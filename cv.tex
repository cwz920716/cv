%%%%%%%%%%%%%%%%%%%%%%%%%%%%%%%%%%%%%%%%%
% Medium Length Graduate Curriculum Vitae
% LaTeX Template
% Version 1.1 (9/12/12)
%
% This template has been downloaded from:
% http://www.LaTeXTemplates.com
%
% Original author:
% Rensselaer Polytechnic Institute (http://www.rpi.edu/dept/arc/training/latex/resumes/)
%
% Important note:
% This template requires the res.cls file to be in the same directory as the
% .tex file. The res.cls file provides the resume style used for structuring the
% document.
%
%%%%%%%%%%%%%%%%%%%%%%%%%%%%%%%%%%%%%%%%%

%----------------------------------------------------------------------------------------
%	PACKAGES AND OTHER DOCUMENT CONFIGURATIONS
%----------------------------------------------------------------------------------------

\documentclass[margin, 9pt]{res} % Use the res.cls style, the font size can be changed to 11pt or 12pt here

\usepackage[T1]{fontenc}
\usepackage[sc]{mathpazo} % Use the great palatino font provided by the mathpazo package. Add osf for old style figures

\usepackage[colorlinks = true,
            linkcolor = blue,
            urlcolor  = blue,
            citecolor = blue,
            anchorcolor = blue]{hyperref}
\usepackage{color}
\usepackage{parskip}
\usepackage{enumitem}
\usepackage[T1]{fontenc}

\setlength{\textwidth}{5.1in} % Text width of the document

\begin{document}

%----------------------------------------------------------------------------------------
%	NAME AND ADDRESS SECTION
%----------------------------------------------------------------------------------------

\moveleft.5\hoffset\centerline{\huge\bf Wenzhi Cui} % Your name at the top
\vspace{-5pt}
 
\moveleft\hoffset\vbox{\hrule width\resumewidth height .5pt}\smallskip % Horizontal line after name; adjust line thickness by changing the '1pt'
 
%\moveleft.5\hoffset\centerline{UT Austin ECE}
%\moveleft.5\hoffset\centerline{yzhu@utexas.edu}
%\moveleft.5\hoffset\centerline{\color{blue}{\url{http://yuhaozhu.com/}}}

\begin{minipage}{0.5\linewidth}
  \flushleft
  Google LLC \\
\end{minipage}
\begin{minipage}{0.5\linewidth}
  \flushright
  \texttt{cwz920716@gmail.com} \\
\end{minipage}

%----------------------------------------------------------------------------------------

\begin{resume}

%----------------------------------------------------------------------------------------
%	RESEARCH INTERESTS SECTION
%----------------------------------------------------------------------------------------

\vspace*{-5pt}
\section{SUMMARY} 

I am interested in building high-performance, energy-efficient computer systems. My past projects focus on developing advanced compiler and runtime support for emerging programming models and hardware platforms, such as GPGPU and Pixel Visual Core (PVC).

%----------------------------------------------------------------------------------------
%	EDUCATION SECTION
%----------------------------------------------------------------------------------------

%\vspace*{-5pt}
\section{EDUCATION}

\textbf{The University of Texas at Austin} \hfill\textit{2014-2018}\\
M.S., Computer Science\\

\vspace*{-5pt}
\textbf{Nanjing University, Nanjing, China} \hfill\textit{2010-2014}\\
B.S., Software Engineering

%----------------------------------------------------------------------------------------
%	SKILL SECTION
%----------------------------------------------------------------------------------------
 
\section{SKILLS}

\vspace*{-1pt}
\begin{itemize}[leftmargin=*] \itemsep -4pt
	\item Programming Languages: C/C++, Cuda, Java, JavaScript, Python
	\item System Software: Halide, LLVM, Node.js, Coq
	\item Hardware Description Language: Verilog, Bluespec Verilog
\end{itemize}

%----------------------------------------------------------------------------------------
%	INTERN EXPERIENCE SECTION
%----------------------------------------------------------------------------------------

\section{WORK\\ EXPERIENCE}

{\textbf{Google Full-time}} \hfill{06/2018 - present}\\
\vspace*{-10pt}
\begin{itemize}[leftmargin=*] \itemsep -3pt
\vspace*{-5pt}
	\item Compiler/runtime support for new hardware/language features on Pixel Visual Core
	\item Add compiler features to assit application teams in porting emerging applications to PVC
	\item Improve PVC compiler infrastucture reliability and performance via fixing critical bugs and refactoring compiler code base.
\end{itemize} 

\medskip
{\textbf{Graduate Research Assistant (GRA)}} \hfill{2017 - 2018}\\
\vspace*{-10pt}
\begin{itemize}[leftmargin=*] \itemsep -3pt
\vspace*{-5pt}
	\item LLVM based instrumentation Framework for CUDA device kernels and host applications
	\item Profile conditional branch divergence and memory divergence analysis on GPGPU
	\item Profile GPU/CPU/Battery power on drone applications on NVIDIA Tegra TX2
\end{itemize}

\medskip
{\textbf{Google Intern}} \hfill{Summer 2017}\\
\vspace*{-10pt}
\begin{itemize}[leftmargin=*] \itemsep -3pt
\vspace*{-5pt}
	\item Compiler/runtime support for advanced hardware features on Pixel Visual Core (PVC)
	\item Halide library for image applications (Transpose, Rectification, etc.) on PVC
\end{itemize}
 
\medskip
{\textbf{Graduate Research Assistant (GRA)}} \hfill{2016 - 2017}\\
\vspace*{-10pt}
\begin{itemize}[leftmargin=*] \itemsep -3pt
\vspace*{-5pt}
	\item Develope a program profiling technique called the "Event Dependence Graph" (EDG) to deconstruct the server response time of event-driven applications such as Node.js
	\item Use EDG to characterize the tail latency of Node.js application and identified JavaScript garbage collector as a dominant root cause for the Node.js tail
	\item Demonstrate how to alleviate the tail latency by applying frequency boosting during garbage collection and carefully tuning garbage collector parameters
\end{itemize}

\medskip
{\textbf{IBM Research Austin}} \hfill{Summer 2015}\\
\vspace*{-10pt}
\begin{itemize}[leftmargin=*] \itemsep -3pt
\vspace*{-5pt}
	\item Analyze the HTTP request latency distribution of web servers in data centers
	\item Design an end-host based load balancing scheme by offloading HTTP requests
	\item Implemente a prototype in OpenVSwitch
\end{itemize}

%----------------------------------------------------------------------------------------
%	PUBLICATIONS SECTION
%----------------------------------------------------------------------------------------
 
%\vspace*{-5pt}
\section{PUBLICATIONS}

{\large\textbf{Conference Papers}}

\begin{itemize}[leftmargin=*] \itemsep -3pt
%\vspace*{-5pt}
	\item Behzad Boroujerdian, Hasan Genc, Srivatsan Krishnan, Wenzhi Cui, Aleksandra Faust, Vijay Janapa Reddi\\
          \href{https://edge.seas.harvard.edu/files/zad-18-micro-mavbench.pdf}{MAVBench: Micro Aerial Vehicle Benchmarking}\\
          \textit{MICRO 2016}

	\item Nadav Chachmon, Daniel Richins, Robert Cohn, Magnus Christensson, Wenzhi Cui, Vijay Janapa Reddi\\
          \href{http://3nity.io/~vj/downloads/publications/sae16ics.pdf}{Simulation and Analysis Engine for Scale-Out Workloads}\\
          \textit{ICS 2016}

	\item Wenzhi Cui, Chen Qian\\
          \href{http://arxiv.org/pdf/1403.8065.pdf}{Scalable and Load-balanced Data Center Multicast}\\
          \textit{Globecom 2015}

	\item Wenzhi Cui, Chen Qian\\
          \href{http://www.cs.uky.edu/~qian/papers/DiFS.pdf}{DiFS: Distributed Flow Scheduling for Adaptive Routing in Hierarchical Data Center Networks}\\
          \textit{ANCS 2014}
\end{itemize}

{\large\textbf{Patents}}

\begin{itemize}[leftmargin=*] \itemsep -3pt
%\vspace*{-5pt}
	\item Kanak B. Agarwal, Wenzhi Cui, Wesley Felter, Yu Gu, Eric Rozner\\
	  \href{https://patents.google.com/patent/US20180159922A1}{Job assignment using artificially delayed responses in load balanced groups}\\
	\item Kanak B. Agarwal, Wenzhi Cui, Wesley Felter, Yu Gu, Eric Rozner\\
	  \href{https://patents.google.com/patent/US20180157539A1}{Tail latency-based job offloading in load-balanced groups}\\
	\item Kanak B. Agarwal, Wenzhi Cui, Wesley Felter, Yu Gu, Eric Rozner\\
	  \href{https://patents.google.com/patent/US20180159775A1}{Offloading at a virtual switch in a load-balanced group}\\
\end{itemize}

%----------------------------------------------------------------------------------------
%	HONOR SECTION
%----------------------------------------------------------------------------------------
 
\section{HONORS \&\\ RECOGNITIONS}

\vspace*{-1pt}
\begin{itemize}[leftmargin=*] \itemsep -1pt
	\item UT Austin Microelectronics and Computer Development Fellowship, 2014-2017
	\item Google Scholarship, 2013
\end{itemize}

%----------------------------------------------------------------------------------------
%	SELECTED PROJECTS SECTION
%---------------------------------------------------------------------------------------- 

\section{COURSEWORK\\ PROJECTS}

{\large\textbf{Misc.}}

\begin{itemize}[leftmargin=*] \itemsep -3pt
\vspace*{-5pt}
	\item Verify the correctness of (simplified) mark-sweep garbage collector using Coq
	\item Implemente an analysis pass in the LLVM compiler to determine C/C++ pointer bounds and integrate with Softbound, a compiler transformation pass for enforcing spatial safety of C/C++ pointers
	\item \href{http://www.cs.utexas.edu/~wc8348/cs384g/draw.html}{OpenGL Based Cloth Simulation using mass-spring model}
	\item Utilize parallel execution and asynchronous IO to speed up the recursive copy operation on SSD
	\item Implement a multi-cycle DLX (simplified MIPS) microprocessor using Verilog
\end{itemize}

%----------------------------------------------------------------------------------------
%	TEACHING EXPERIENCE SECTION
%----------------------------------------------------------------------------------------

\section{TEACHING\\ EXPERIENCE} 

{\large{Teaching Assistant}}\\
\vspace*{-5pt}
\begin{itemize}[leftmargin=*] \itemsep -2pt
\vspace*{-5pt}
	\item Undergraduate: Programming Languages(Honor), Principles of Computer Systems, Computer Networks
	\item Graduate: Code Generation and Optimization
\end{itemize}

%----------------------------------------------------------------------------------------
%	COURSEWORK SECTION
%----------------------------------------------------------------------------------------

\section{COURSEWORK} 

\vspace*{-2pt}
	Computer Architecture, Compilers, Advanced Operating Systems, Algorithms: Techniques/Theory, Formal Verification and Semantics, Natural Language Processing, Program Verification, Hardware Verification, Programming Languages, Computer Graphics, Physical Simulation

%----------------------------------------------------------------------------------------

\end{resume}
\end{document}
