%%%%%%%%%%%%%%%%%%%%%%%%%%%%%%%%%%%%%%%%%
% Medium Length Graduate Curriculum Vitae
% LaTeX Template
% Version 1.1 (9/12/12)
%
% This template has been downloaded from:
% http://www.LaTeXTemplates.com
%
% Original author:
% Rensselaer Polytechnic Institute (http://www.rpi.edu/dept/arc/training/latex/resumes/)
%
% Important note:
% This template requires the res.cls file to be in the same directory as the
% .tex file. The res.cls file provides the resume style used for structuring the
% document.
%
%%%%%%%%%%%%%%%%%%%%%%%%%%%%%%%%%%%%%%%%%

%----------------------------------------------------------------------------------------
%	PACKAGES AND OTHER DOCUMENT CONFIGURATIONS
%----------------------------------------------------------------------------------------

\documentclass[margin, 9pt]{res} % Use the res.cls style, the font size can be changed to 11pt or 12pt here

\usepackage[T1]{fontenc}
\usepackage[sc]{mathpazo} % Use the great palatino font provided by the mathpazo package. Add osf for old style figures

\usepackage[colorlinks = true,
            linkcolor = blue,
            urlcolor  = blue,
            citecolor = blue,
            anchorcolor = blue]{hyperref}
\usepackage{color}
\usepackage{parskip}
\usepackage{enumitem}
\usepackage[T1]{fontenc}

\setlength{\textwidth}{5.1in} % Text width of the document

\begin{document}

%----------------------------------------------------------------------------------------
%	NAME AND ADDRESS SECTION
%----------------------------------------------------------------------------------------

\moveleft.5\hoffset\centerline{\huge\bf Wenzhi Cui} % Your name at the top
\vspace{-5pt}
 
\moveleft\hoffset\vbox{\hrule width\resumewidth height .5pt}\smallskip % Horizontal line after name; adjust line thickness by changing the '1pt'
 
%\moveleft.5\hoffset\centerline{UT Austin ECE}
%\moveleft.5\hoffset\centerline{yzhu@utexas.edu}
%\moveleft.5\hoffset\centerline{\color{blue}{\url{http://yuhaozhu.com/}}}

\begin{minipage}{0.5\linewidth}
  \flushleft
  The University of Texas at Austin \\
  Computer Science\\
\end{minipage}
\begin{minipage}{0.5\linewidth}
  \flushright
  \texttt{wc8348@cs.utexas.edu} \\
\end{minipage}

%----------------------------------------------------------------------------------------

\begin{resume}

%----------------------------------------------------------------------------------------
%	RESEARCH INTERESTS SECTION
%----------------------------------------------------------------------------------------

\vspace*{-5pt}
\section{RESEARCH\\ INTERESTS} 

I am interested in building high-performance, energy-efficient computer systems. My past projects focus on developing advanced compiler and runtime support for emerging programming models and hardware platforms, such as GPGPU and Node.js.

%----------------------------------------------------------------------------------------
%	EDUCATION SECTION
%----------------------------------------------------------------------------------------

%\vspace*{-5pt}
\section{EDUCATION}

\textbf{The University of Texas at Austin} \hfill\textit{2014-present}\\
M.S. student, Computer Science\\
Expected Graduation Date: 2018-05\\

\vspace*{-5pt}
\textbf{Nanjing University, Nanjing, China} \hfill\textit{2010-2014}\\
B.S., Software Engineering

%----------------------------------------------------------------------------------------
%	SKILL SECTION
%----------------------------------------------------------------------------------------
 
\section{SKILLS}

\vspace*{-1pt}
\begin{itemize}[leftmargin=*] \itemsep -4pt
	\item Programming Languages: C/C++, Cuda, Java, JavaScript, Python
	\item System Software: LLVM, Node.js, Halide, Coq
	\item Hardware Description Language: Verilog, Bluespec Verilog
\end{itemize}

%----------------------------------------------------------------------------------------
%	INTERN EXPERIENCE SECTION
%----------------------------------------------------------------------------------------

\section{INTERN\\ EXPERIENCE}

{\textbf{Google}} \hfill{Summer 2018}\\
\vspace*{-10pt}
\begin{itemize}[leftmargin=*] \itemsep -3pt
\vspace*{-5pt}
	\item Compiler Support for consumer hardware on mobile platform
  \item Runtime Extension for Image Processing Applications in Halide
\end{itemize} 

\medskip
{\textbf{IBM Research Austin}} \hfill{Summer 2015}\\
\vspace*{-10pt}
\begin{itemize}[leftmargin=*] \itemsep -3pt
\vspace*{-5pt}
	\item Analyzed the HTTP request latency distribution of web servers in data centers
	\item Proposed an end-host based load balancing scheme by offloading HTTP requests
	\item Implemented a prototype in OpenVSwitch
\end{itemize}

\medskip
{\textbf{University of Queensland}} \hfill{Winter 2012}\\
\vspace*{-10pt}
\begin{itemize}[leftmargin=*] \itemsep -3pt
\vspace*{-5pt}
	\item Studied the query optimizer of MongoDB database
	\item Tweaked MongoDB Query Optimizer to achieve better performance 
\end{itemize}

%----------------------------------------------------------------------------------------
%	PUBLICATIONS SECTION
%----------------------------------------------------------------------------------------
 
%\vspace*{-5pt}
\section{PUBLICATIONS}

{\large\textbf{Conference Papers}}

\begin{itemize}[leftmargin=*] \itemsep 0pt
%\vspace*{-5pt}
  \item Nadav Chachmon, Daniel Richins, Robert Cohn, Magnus Christensson, Wenzhi Cui, Vijay Janapa Reddi
          \href{http://3nity.io/~vj/downloads/publications/sae16ics.pdf}{Simulation and Analysis Engine for Scale-Out Workloads}\\
          \textit{ICS 2016}

	\item Wenzhi Cui, Chen Qian\\
          \href{http://arxiv.org/pdf/1403.8065.pdf}{Scalable and Load-balanced Data Center Multicast}\\
          \textit{Globecom 2015}

	\item Wenzhi Cui, Chen Qian\\
          \href{http://www.cs.uky.edu/~qian/papers/DiFS.pdf}{DiFS: Distributed Flow Scheduling for Adaptive Routing in Hierarchical Data Center Networks}\\
          \textit{ANCS 2014}
\end{itemize}

%----------------------------------------------------------------------------------------
%	HONOR SECTION
%----------------------------------------------------------------------------------------
 
\section{HONORS \&\\ RECOGNITIONS}

\vspace*{-1pt}
\begin{itemize}[leftmargin=*] \itemsep -1pt
	\item UT Austin Microelectronics and Computer Development Fellowship, 2014-2017
	\item Google Scholarship, 2013
\end{itemize}

%----------------------------------------------------------------------------------------
%	SELECTED PROJECTS SECTION
%---------------------------------------------------------------------------------------- 

\section{SELECTED\\ PROJECTS}

{\large\textbf{The University of Texas at Austin (\textit{Graduate})}}\\

\vspace*{-7pt}
 LLVM-based Instrumentationa and Profiling for CUDA programs\\
\vspace*{-10pt}
\begin{itemize}[leftmargin=*] \itemsep -3pt
\vspace*{-5pt}
	\item Instrumentation Framework for CUDA device kernels and host applications
  \item Profiling tools for conditional branch divergence and memory divergence analysis in the runtime
  \item (Ongoing) Perform launch time optimizations such as constant propagation, loop unrolling, etc.
\end{itemize}

\medskip
\vspace*{-7pt}
\href{http://www.cs.utexas.edu/~wc8348/cs384g/draw.html}{OpenGL Based Cloth Simulation}\\
\vspace*{-10pt}
\begin{itemize}[leftmargin=*] \itemsep -3pt
\vspace*{-5pt}
	\item Implemented a mass-spring model to simulate cloth dynamics, such as partical interation and damping forces
	\item Simulated the interaction between cloth and rigid body objects. We implemented box, sphere and triangle in our scene and created a collision detection method for each of the object.
	\item Used OpenGL to render the scene in real-time. Video demos are available \href{https://www.youtube.com/playlist?list=PLhg5e7MedlDPsw6D1Lg_bPupiMRJJ9fnm&disable_polymer=true}{online}.
\end{itemize}

\medskip
\vspace*{-7pt}
Deconstructing Tail Latencies with Event-Dependence Graphs in Managed Runtimes (Node.js)\\
\vspace*{-10pt}
\begin{itemize}[leftmargin=*] \itemsep -3pt
\vspace*{-5pt}
	\item Developed a program profiling technique called the "Event Dependence Graph" (EDG) to deconstruct the server response time of event-driven applications such as Node.js
	\item Used EDG to characterize the tail latency of Node.js application and identified JavaScript garbage collector as a dominant root cause for the Node.js tail
	\item Demonstrated how to alleviate the tail latency by applying frequency boosting during garbage collection and carefully tuning garbage collector parameters
\end{itemize}

\medskip
Static Analysis on C/C++ Pointers Bounds Checking\\
\vspace*{-10pt}
\begin{itemize}[leftmargin=*] \itemsep -3pt
\vspace*{-5pt}
	\item Designed and implemented an analysis pass in the LLVM compiler to determine C/C++ pointer bounds at compile time
	\item Integrated static analysis with Softbound, a compile-time transformation for enforcing spatial safety of C/C++ pointers, to balance performance and coverage
\end{itemize}

\medskip
Parallel Asynchronous Recursive Copy on SSD\\
\vspace*{-10pt}
\begin{itemize}[leftmargin=*] \itemsep -3pt
\vspace*{-5pt}
	\item Characterized the performance of copy command in the Linux Shell for large-volume directories on Solid State Drive
	\item Implemented a prototype which utilizes parallel execution and asynchronous IO to speed up the recursive copy operation
\end{itemize}

\bigskip
{\large\textbf{Nanjing University (\textit{Undergraduate})}}\\

\vspace*{-7pt}
DLX Instruction Set Prototype on FPGA \hfill(\textit{03/2013-08/2013})\\
\vspace*{-10pt}
\begin{itemize}[leftmargin=*] \itemsep -3pt
\vspace*{-5pt}
	\item Designed a multi-cycle DLX (simplified MIPS) microprocessor using Verilog
	\item Extended MMU and added circuit support for connecting keyboard and monitor
\end{itemize}

%----------------------------------------------------------------------------------------
%	TEACHING EXPERIENCE SECTION
%----------------------------------------------------------------------------------------

\section{TEACHING\\ EXPERIENCE} 

{\large{Teaching Assistant}}\\
\vspace*{-5pt}
\begin{itemize}[leftmargin=*] \itemsep -2pt
\vspace*{-5pt}
	\item Upper Division, Programming Languages: Honor
	\item Upper Division, Principles of Computer Systems
	\item Upper Division, Computer Networks
	\item Graduate, Code Generation and Optimization
\end{itemize}

%----------------------------------------------------------------------------------------
%	COURSEWORK SECTION
%----------------------------------------------------------------------------------------

\section{COURSEWORK} 

\vspace*{-2pt}
\begin{itemize}[leftmargin=*] \itemsep -2pt
	\item Computer Architecture (with Prof. Yale N. Patt)
	\item Compilers (with Prof. Calvin Lin)
	\item Advanced Operating Systems (with Prof. Emmett Witchel)
	\item Algorithms: Techniques/Theory (with Prof. Vijaya Ramachandran)
	\item Formal Verification and Semantics (with Prof. E. A. Emerson)
	\item Natural Language Processing (with Prof. Raymond J. Mooney)
	\item Program Verification (with Prof. Thomas Dillig)
	\item Programming Languages (with Prof. William Cook)
	\item Computer Graphics (with Prof. Donald S. Fussell)
\end{itemize}

%----------------------------------------------------------------------------------------

\end{resume}
\end{document}
